%%%%%%%%%%%%%%%%%%%%%%%%%%%%%%%%%%%%%%%%%%%%%%%%%%%%%%%%%%%%%%%%%%%%%%%%%%%%%%%%
\intro
%%%%%%%%%%%%%%%%%%%%%%%%%%%%%%%%%%%%%%%%%%%%%%%%%%%%%%%%%%%%%%%%%%%%%%%%%%%%%%%%

В наше время создается огромное количество различных библиотек и систем, которые не всегда получают должную документацию. Надежность и правильность таких систем можно обеспечить за счет определенных спецификаций. 
Однако чаще всего спецификации отсутствуют, так как создание их довольно трудоемкое дело. Плюс ко всему существует ряд экспериментальных инструментов, работающих со спецификациями.
Основная цель ВКР – создать инструмент, который поможет разработчику генерировать каркас описания, используя LibSL спецификацию.
В данной работе будут рассмотрены практические результаты.
