%%%%%%%%%%%%%%%%%%%%%%%%%%%%%%%%%%%%%%%%%%%%%%%%%%%%%%%%%%%%%%%%%%%%%%%%%%%%%%%%
\intro
%%%%%%%%%%%%%%%%%%%%%%%%%%%%%%%%%%%%%%%%%%%%%%%%%%%%%%%%%%%%%%%%%%%%%%%%%%%%%%%%

В наше время создается огромное количество различных библиотек и систем, а сроки на создание продуктов снизиль с нескольких месяцев до нескольких недель.
В связи с чем разработчики стремятся ускорить развертывание нового кода или новых модулей в производстве для своих систем при почти полном отсутствии контроля качества разрабатываемого ПО.
Для того чтобы разработать качественное программное обеспечение, разработчикам необходимо достичь полное соответствие функциональным (явным) и нефункциональным требованиям для разрабатываемой системы.
Характеристики функциональных требований \cite{kursMAOK_1}:
%
\begin{itemize*}
\item Пригодность (соответствие требуемому набору функций);
\item Корректность, правильность, точность;
\item Соответствие стандартам ;
\item Защищенность;
\end{itemize*}
%

Как правило, функциональные требования имеют специфику для конкретной системы и относятся к конкретной ее части, а не к системе целиком и гарантируют то, что система не противоречит поставленной цели.
Нефункциональные требования же описывают всю систему, то есть касаются всего продукта.

Существуют определенные методы обеспечения качества ПО, применение которых помогут разработчикам в достижении высокого качества программного обеспечения.
Методы обеспечения качества, при их должном соблюдении во время процесса разработки, могут гарантировать определенного результата, который положительно отражается на качестве ПО.

Определим некоторые основные методы обеспечения качества ПО \cite{PI_book}:
%
\begin{itemize*}
\item Тестирование;
\item Профилирование;
\item Мониторинг;
\item Анализ трасс исполнения;
\item Рефакторинг программ;
\item Метод проверки модели;
\item Аудит программного кода;
\item Статический анализ кода (требования, спецификации, исходный код программы);
\end{itemize*}

Главным объектом анализа данной работы является метод опеспечения качества ПО, основанный на спецификациях.
Такие методы гарантируют надежность и качество систем за счет определенных спецификаций.
С помощью спецификации мы можем получить исчерпывающее описание программного продукта, как конечного, так и разрабатываемого, включая его назначение, основные бизнес-процессы, которые будут поддерживаться, функции, ключевые параметры производительности и поведение.
Оданако эти методы имеют явные проблемы, так чаще всего спецификации отсутсвуют, и неясно откуда их взять или же как построить, так как это очень сложный и довольно трудоемкий процесс.
Плюс ко всему существует ряд экспериментальных инструментов, работающих со спецификациями.
Поэтому нужны подходы, которые бы позволяли частично автоматизировать процесс построения спецификаций.

Основная цель ВКР – создать инструмент, который поможет разработчику генерировать основную базу описания разрабатываемого ПО. Такой инструмент сократит время, затрачиваемое на написание полной спецификации библиотеки или системы, так как позволит создать необходимый каркас спецификации, наполненный дополнительной информацией.
Работа состоит из пяти разделов. В первом разделе проанализированы подходы к генерации спецификации:
%
\begin{itemize*}
\item Извлечение автоматов и списка функций(структра библиотеки)
\item Извлечение контрактов(предусловий и постусловий) для каждой функции
\item Извлечение окружения состояния автоматов и переходов автоматов
\end{itemize*}
%

Также приведены подходы к статическому анализу кода и существующие инструменты для выполнения этой задачи. Во втором разделе сформулированы требования к созданному инструменту автоматизированной генерации формальной спецификации библиотек. Третий раздел описывает архитектуру инструмента. А в четвертом и пятом разделах будут описаны разработка и тестирование инструмента соответственно.
