%%%%%%%%%%%%%%%%%%%%%%%%%%%%%%%%%%%%%%%%%%%%%%%%%%%%%%%%%%%%%%%%%%%%%%%%%%%%%%%%
\intro
%%%%%%%%%%%%%%%%%%%%%%%%%%%%%%%%%%%%%%%%%%%%%%%%%%%%%%%%%%%%%%%%%%%%%%%%%%%%%%%%

В наше время создается огромное количество различных библиотек и систем, которые не всегда получают должную документацию. Надежность и правильность таких систем можно обеспечить за счет определенных спецификаций.
Однако чаще всего спецификации отсутствуют, так как создание их довольно трудоемкое дело. Плюс ко всему существует ряд экспериментальных инструментов, работающих со спецификациями.
Основная цель ВКР – создать инструмент, который поможет разработчику генерировать каркас описания, используя LibSL спецификацию. Такой инструмент сократит время, затрачиваемое на написание полной спецификации библиотеки или системы, так как позволит создать необходимый каркас спецификации, наполненный дополнительной информацией, которую возможно извлечь из исходного кода.
Работа состоит из пяти разделов. В первом разделе проанализированы подходы к статическому анализу кода и существующие инструменты для выполнения этой задачи. Во втором разделе сформулированы требования к созданному инструменту автоматизированной генерации формальной спецификации библиотек. Третий раздел описывает архитектуру инструмента. А в четвертом и пятом разделах будут описаны разработка и тестирование инструмента соответственно.
