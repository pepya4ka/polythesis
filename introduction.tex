%%%%%%%%%%%%%%%%%%%%%%%%%%%%%%%%%%%%%%%%%%%%%%%%%%%%%%%%%%%%%%%%%%%%%%%%%%%%%%%%
\intro
%%%%%%%%%%%%%%%%%%%%%%%%%%%%%%%%%%%%%%%%%%%%%%%%%%%%%%%%%%%%%%%%%%%%%%%%%%%%%%%%

В наше время создается огромное количество различных библиотек и систем, а сроки на создание продуктов снизиль с нескольких месяцев до нескольких недель.
В связи с чем разработчики больше стремятся на развертывании нового кода или новых модулей в производство для своих систем при почти полном отсутствии контроля качества разрабатываемого ПО.
Для того чтобы разработать качественное программное обеспечение, разработчикам необходимо достичь полное соответствие функциональным (явным) и нефункциональным требованиям для разрабатываемой системы.
Функциональные требования можно описать следующим образом:
%
\begin{itemize*}
\item Понятия "что" делает приложение;
\item Основные требования;
\item Поведение отдельной части приложения;
\item Функции приложения были протестированы;
\end{itemize*}
%

В свою очередь нефункиональные описываются:
%
\begin{itemize*}
\item Понятия "как" приложение что-то делает;
\item Второстепенные приложения требования;
\item Поведение всего приложения;
\item Гарантии удобства использования, надежности и производительности;
\end{itemize*}
%

Как правило, функциональные требования имеют специфику для конкретной системы и относятся к конкретной ее части, а не к системе целиком и гарантируют то, что система не противоречит поставленной цели.
Нефункциональные требования же описывают всю систему, то есть касаются всего продукта.

Существуют определенные методы обеспечения качества ПО, применение которых помогут разработчикам в достижении высокого качества программного обеспечения.
Методы обеспечения качества, при их должном соблюдении во время процесса разработки, могут гарантировать определенного результата, который положительно отражается на качестве ПО.

Определим некоторые основные методы обеспечения качества ПО:
%
\begin{itemize*}
\item Автоматизированное модульное, системное или нагрузочное тестирование;
\item Интеграционное тестирование и другие виды ручного тестирования;
\item Покрытие кода unit-тестами;
\item Статический анализ кода (требования, спецификации, исходный код программы);
\item Методы формальной верификации;
\item Гибридные методы;
\end{itemize*}
%

Главым объектом анализа данной работы является метод опеспечения качества ПО, основанные на спецификациях.
Такие методы гарантируют надежность и качество систем за счет определенных спецификаций.
С помощью спецификации мы можем получить исчерпывающее описание программного продукта, как конечного, так и разрабатываемого, включая его назначение, основные бизнес-процессы, которые будут поддерживаться, функции, ключевые параметры производительности и поведение.
Оданако эти методы имеют явные проблемы, так чаще всего спецификации отсутсвуют, и неясно откудать их взять или же как построить спецификацию, так как это очень сложный и довольно трудоемкий процесс.
Плюс ко всему существует ряд экспериментальных инструментов, работающих со спецификациями.
Поэтому нужны подходы, которые бы позволяли частично автоматизировать процесс построения спецификаций.

Основная цель ВКР – создать инструмент, который поможет разработчику генерировать основную базу описания разрабатываемой библиотеки. Такой инструмент сократит время, затрачиваемое на написание полной спецификации библиотеки или системы, так как позволит создать необходимый каркас спецификации, наполненный дополнительной информацией.
Работа состоит из пяти разделов. В первом разделе проанализированы подходы к генерации спецификации:
%
\begin{itemize*}
\item Извлечение автоматов и списка функций(структра библиотеки)
\item Извлечение контрактов(предусловий и постусловий) для каждой функции
\item Извлечение окружения состояния автоматов и переходов автоматов
\end{itemize*}
%

Также приведены подходы к статическому анализу кода и существующие инструменты для выполнения этой задачи. Во втором разделе сформулированы требования к созданному инструменту автоматизированной генерации формальной спецификации библиотек. Третий раздел описывает архитектуру инструмента. А в четвертом и пятом разделах будут описаны разработка и тестирование инструмента соответственно.
