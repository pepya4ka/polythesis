%%%%%%%%%%%%%%%%%%%%%%%%%%%%%%%%%%%%%%%%%%%%%%%%%%%%%%%%%%%%%%%%%%%%%%%%%%%%%%%%
\chapter{Анализ подходов к генерации спецификации}
%%%%%%%%%%%%%%%%%%%%%%%%%%%%%%%%%%%%%%%%%%%%%%%%%%%%%%%%%%%%%%%%%%%%%%%%%%%%%%%%

В данном разделе рассматриваются существующие подходы к автоматизированной генерации спецификаций.

%%%%%%%%%%%%%%%%%%%%%%%%%%%%%%%%%%%%%%%%%%%%%%%%%%%%%%%%%%%%%%%%%%%%%%%%%%%%%%%%
\section{Подходы к генерации спецификациии}
%%%%%%%%%%%%%%%%%%%%%%%%%%%%%%%%%%%%%%%%%%%%%%%%%%%%%%%%%%%%%%%%%%%%%%%%%%%%%%%%

Ниже перечислены основные подходы к генерации формальной спецификации библиотек:
%
\begin{itemize*}
\item ???Извлечение автоматов и списка функций(структра библиотеки)
\item Извлечение контрактов(предусловий и постусловий) для каждой функции
\item Извлечение окружения состояния автоматов и переходов автоматов
\end{itemize*}
%

Чаще всего существующие подходы можно отнести к статическому типу. Но также можно встретить и динамические подходы ивзлечения контрактов функции, но это, как следует из практики, негативно сказывается на производительности.

%%%%%%%%%%%%%%%%%%%%%%%%%%%%%%%%%%%%%%%%%%%%%%%%%%%%%%%%%%%%%%%%%%%%%%%%%%%%%%%%
\section{???Извлечение автоматов и списка функций}
%%%%%%%%%%%%%%%%%%%%%%%%%%%%%%%%%%%%%%%%%%%%%%%%%%%%%%%%%%%%%%%%%%%%%%%%%%%%%%%%

Данный подход является основополагающим в генерации спецификации

%%%%%%%%%%%%%%%%%%%%%%%%%%%%%%%%%%%%%%%%%%%%%%%%%%%%%%%%%%%%%%%%%%%%%%%%%%%%%%%%
\section{Извлечение предусловий и постусловий для функции}
%%%%%%%%%%%%%%%%%%%%%%%%%%%%%%%%%%%%%%%%%%%%%%%%%%%%%%%%%%%%%%%%%%%%%%%%%%%%%%%%

Один из вариантов, с помощью которого можно дополнить сгенерированный каркас спецификации, то есть структуру библиотеки, это детектор инвариантов. Такие детекторы помогают но основе исходного кода библиотеки генерировать постусловия и предусловия для каждой отдельной функции в классах библиотеки.

Один из примеров реализации данного подхода является система Daikon\cite{Детектор90:online}

%%%%%%%%%%%%%%%%%%%%%%%%%%%%%%%%%%%%%%%%%%%%%%%%%%%%%%%%%%%%%%%%%%%%%%%%%%%%%%%%
\section{Извлечение окружения состояния автоматов и переходов автоматов}
%%%%%%%%%%%%%%%%%%%%%%%%%%%%%%%%%%%%%%%%%%%%%%%%%%%%%%%%%%%%%%%%%%%%%%%%%%%%%%%%
