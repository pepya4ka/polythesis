%%%%%%%%%%%%%%%%%%%%%%%%%%%%%%%%%%%%%%%%%%%%%%%%%%%%%%%%%%%%%%%%%%%%%%%%%%%%%%%%
\chapter{Проектирование автоматизированной системы генерации формальной спецификации библиотек}
%%%%%%%%%%%%%%%%%%%%%%%%%%%%%%%%%%%%%%%%%%%%%%%%%%%%%%%%%%%%%%%%%%%%%%%%%%%%%%%%

В данном разделе описывается процесс разработки инструмента генерации формальной спецификации библиотек.

%%%%%%%%%%%%%%%%%%%%%%%%%%%%%%%%%%%%%%%%%%%%%%%%%%%%%%%%%%%%%%%%%%%%%%%%%%%%%%%%
\section{Анализ библиотеки}
%%%%%%%%%%%%%%%%%%%%%%%%%%%%%%%%%%%%%%%%%%%%%%%%%%%%%%%%%%%%%%%%%%%%%%%%%%%%%%%%

В таблице~\ref{tabular:first_tab} представлены элементы спецификации и способ их извлечения:

\begin{table}[H]
	\caption{Элементы спецификации и способы их извлечения}
	\begin{center}
		\begin{tabular}{|l|l|l|}
			\hline
			& Элемент спецификации & Способ извлечения\\ \hline

			1 & Заголовок библиотеки & Статический анализ java или jar файлов\\ \hline
			2 & Описание о подключаемых файлов или пакетов & Статический анализ java или jar файлов\\ \hline
			3 & Описание псевдонимов типов & Статический анализ java или jar файлов\\ \hline
			4 & Описание автоматов & Статический анализ java или jar файлов\\ \hline
			5 & Описание функций API библиотеки & Статический анализ java или jar файлов\\ \hline
			6 & Влияние на окружении функции & Легковесный анализ тела функции\\ \hline
		    7 & Предусловия и постусловия функции & Серьезный статический анализ тела функции для построения суммаризации\\ \hline
			8 & Поведение библиотеки на уровне автоматов & Парсинг публичных репозиториев с целью извлечения сценариев поведения\\ \hline
		\end{tabular}
		\label{tabular:first_tab}
	\end{center}
\end{table}

В рамках данной работы мы реализуем пункты с 1 по 6 из таблицы~\ref{tabular:first_tab}.
7 и 8 пункты остаются за рамками данной магистерской диссертации, так как предусловия и постусловия достаточно сложная задача.
Для пункта 8 также отдельная задача для парсинга, например, репозиториев с GitHub.

Для того, чтобы процесс проектирования и разработки инструмента шел правильно, нужно сформулировать порядок выполнения задач.
Необходимо понимать, что программный код библиотек не всегда доступен в виде исходного кода, поэтому разрабатываемая система должна принимать на вход исходные данные библиотеки как в виде исходного текста, а именно директории с файлами java, так и java-архим, то есть jar-файл библиотеки.
На основе входных данных и будет извлекаться вся необходимая информация исходной библиотеки, а уже имея каркас спецификации, то есть структуру библиотеки на языке LibSL, она будет наполняться разработанным методом извлечения аннотаций.
В соответствующую аннотацию на языке LibSL будет записываться извлекаемая информация.

%%%%%%%%%%%%%%%%%%%%%%%%%%%%%%%%%%%%%%%%%%%%%%%%%%%%%%%%%%%%%%%%%%%%%%%%%%%%%%%%
\section{Извлечение структуры библиотеки}
%%%%%%%%%%%%%%%%%%%%%%%%%%%%%%%%%%%%%%%%%%%%%%%%%%%%%%%%%%%%%%%%%%%%%%%%%%%%%%%%

Целью этого шага является извлечение структуры из исходной библиотеки.

Способ получения структуры библиотеки можно описать в виде анализа исходного кода или bytecode, а именно описание классов, соответствующих создаваемым объектам библиотеки, описание псевдонимов типов, используемых в библиотеке, описание функций API библиотеки, список импортируемых внешних библиотек в исходной библиотеке.

%%%%%%%%%%%%%%%%%%%%%%%%%%%%%%%%%%%%%%%%
\subsection{Заголовок библиотеки}
%%%%%%%%%%%%%%%%%%%%%%%%%%%%%%%%%%%%%%%%

Для получения заголовка библиотеки необходимо распарсить java файл, или class-файл для jar, и проанализировать пакет в котором они лежат.
На полученной информации можно будет заполнить заголовок библиотеки.

Для jar-файлов также существует вариант получения заголовка библиотеки. Необходимо проанализировать \textit{MANIFEST.MF} файл для получения, например, версии библиотеки.

%%%%%%%%%%%%%%%%%%%%%%%%%%%%%%%%%%%%%%%%
\subsection{Описание о подключаемых файлах или пакетов}
%%%%%%%%%%%%%%%%%%%%%%%%%%%%%%%%%%%%%%%%

Для получения описания о подключаемых файлов или пакетов необходимо также распарсить полученный на вход файл (.java или .class) и проанализировать весь список подключаемых файлов и пакетов.

%%%%%%%%%%%%%%%%%%%%%%%%%%%%%%%%%%%%%%%%
\subsection{Описание псевдонимов типов}
%%%%%%%%%%%%%%%%%%%%%%%%%%%%%%%%%%%%%%%%

Для получения описания псевдонимов типов необходимо проанализировать каждый класс, входящий в исходную библиотеку, и добавить в описание анализируемый класс.
Также необходимо учитывать возвращаемые значения функций и аргументы этих функций, типы которых могут являться классами из другой библиотеки (например, из стандартой библиотеки Java).

%%%%%%%%%%%%%%%%%%%%%%%%%%%%%%%%%%%%%%%%
\subsection{Описание автоматов}
%%%%%%%%%%%%%%%%%%%%%%%%%%%%%%%%%%%%%%%%

Для получения описания автоматов необходимо учитывать каждый класс исходной библиотеки.
Названием класса автомата и типом автомата будет являться название самого анализируемого класса.

%%%%%%%%%%%%%%%%%%%%%%%%%%%%%%%%%%%%%%%%
\subsection{Описание функций API}
%%%%%%%%%%%%%%%%%%%%%%%%%%%%%%%%%%%%%%%%

Для получения описания функций API библиотеки, необходимо распарсить каждую публчиную функцию анализируемого класса и проанализировать ее имя, аргументы, типы аргументов и возвращаемый тип, то есть всю сигнатуру функции.

%%%%%%%%%%%%%%%%%%%%%%%%%%%%%%%%%%%%%%%%%%%%%%%%%%%%%%%%%%%%%%%%%%%%%%%%%%%%%%%%
\section{Влияние на окружении функции}
%%%%%%%%%%%%%%%%%%%%%%%%%%%%%%%%%%%%%%%%%%%%%%%%%%%%%%%%%%%%%%%%%%%%%%%%%%%%%%%%

Главная часть разработанного инструмента в ходе данной магистерской работы является разработка метода извлечения аннотаций.
Для получения окружения функции, то есть влияния функции на окружающую среду, необходимо для каждой публичной функции анализируемого класса провести легковесный статический анализ тела функции.
Результатом анализа наполнить полученную структуру библиотеки дополнительной информацией, а именно сгенерировать аннотацию assigns.

Как было сказано в прошлой главе, существует модели статического анализа, с помощью которых и можно получить необходимые аннотации:
%
\begin{itemize*}
\item Анализ графа потока управления (CFG);
\item Анализ абстрактного синтаксического дерева (AST);
\end{itemize*}
%
К преимуществам статического анализа можно отнести достаточно высокую полноту.
В данном работе было решено остановиться на статическом анализе.

%%%%%%%%%%%%%%%%%%%%%%%%%%%%%%%%%%%%%%%%%%%%%%%%%%%%%%%%%%%%%%%%%%%%%%%%%%%%%%%%
\section{Генерация спецификации}
%%%%%%%%%%%%%%%%%%%%%%%%%%%%%%%%%%%%%%%%%%%%%%%%%%%%%%%%%%%%%%%%%%%%%%%%%%%%%%%%

Непосредственно генерация спецификации является заключительным процессом, во время которого все созданные объекты, которые были извлечены из исходной библиотеки и наполнены, собираются в один файл спецификации.
На данном этапе важно придерживаться определенных правил языка LibSL для того, чтобы генерировать качественную и удобочитаемую для программиста спецификацию.

%%%%%%%%%%%%%%%%%%%%%%%%%%%%%%%%%%%%%%%%%%%%%%%%%%%%%%%%%%%%%%%%%%%%%%%%%%%%%%%%
\section{Выводы}
%%%%%%%%%%%%%%%%%%%%%%%%%%%%%%%%%%%%%%%%%%%%%%%%%%%%%%%%%%%%%%%%%%%%%%%%%%%%%%%%

В данном разделе был рассмотрен процесс разработки инструмента генерации формальной спецификации. Этап разработки был разбит на три основных этапа, которые обязательны для выполнения в процессе генерации спецификации.
В совокупности разработанные методы служат в качестве прототипа инструмента генерации спецификации библиотек.