%%%%%%%%%%%%%%%%%%%%%%%%%%%%%%%%%%%%%%%%%%%%%%%%%%%%%%%%%%%%%%%%%%%%%%%%%%%%%%%%
\chapter{Проектирование автоматизированной системы генерации формальной спецификации библиотек}
%%%%%%%%%%%%%%%%%%%%%%%%%%%%%%%%%%%%%%%%%%%%%%%%%%%%%%%%%%%%%%%%%%%%%%%%%%%%%%%%

В данном разделе описывается процесс разработки инструмента генерации формальной спецификации библиотек.

%%%%%%%%%%%%%%%%%%%%%%%%%%%%%%%%%%%%%%%%%%%%%%%%%%%%%%%%%%%%%%%%%%%%%%%%%%%%%%%%
\section{Анализ библиотеки}
%%%%%%%%%%%%%%%%%%%%%%%%%%%%%%%%%%%%%%%%%%%%%%%%%%%%%%%%%%%%%%%%%%%%%%%%%%%%%%%%

Для того, чтобы процесс проектирования и разработки инструмента шел правильно, нужно сформулировать порядок выполнения задач.
Необходимо понимать, что программный код библиотек не всегда доступен в виде исходного кода, поэтому разрабатываемая система должна принимать на вход исходные данные библиотеки как в виде исходного текста, а именно директории с файлами java, так и java-архим, то есть jar-файл библиотеки.
На основе входных данных и будет извлекаться вся необходимая информация исходной библиотеки, а уже имея каркас спецификации, то есть структуру библиотеки на языке LibSL, она будет наполняться разработанным методом извлечения аннотаций.
В соответствующую аннотацию на языке LibSL будет записываться извлекаемая информация.

%%%%%%%%%%%%%%%%%%%%%%%%%%%%%%%%%%%%%%%%%%%%%%%%%%%%%%%%%%%%%%%%%%%%%%%%%%%%%%%%
\section{Извлечение структуры библиотеки}
%%%%%%%%%%%%%%%%%%%%%%%%%%%%%%%%%%%%%%%%%%%%%%%%%%%%%%%%%%%%%%%%%%%%%%%%%%%%%%%%

Целью этого шага является извлечение структуры из исходной библиотеки.

Способ получения структуры библиотеки можно описать в виде анализа исходного кода или bytecode, а именно описание классов, соответствующих создаваемым объектам библиотеки, описание псевдонимов типов, используемых в библиотеке, описание функций API библиотеки, список импортируемых внешних библиотек в исходной библиотеке.

%%%%%%%%%%%%%%%%%%%%%%%%%%%%%%%%%%%%%%%%%%%%%%%%%%%%%%%%%%%%%%%%%%%%%%%%%%%%%%%%
\section{Извлечение аннотаций}
%%%%%%%%%%%%%%%%%%%%%%%%%%%%%%%%%%%%%%%%%%%%%%%%%%%%%%%%%%%%%%%%%%%%%%%%%%%%%%%%

Главная часть разработанного инструмента в ходе данной магистерской работы является разработка метода извлечения аннотаций.
Задача этого шага, наполнить полученную структуру библиотеки дополнительной информацией, а именно сгенерировать такие аннотации, которые описывают влияние функции на окружающую среду.

Как было сказано в прошлой главе, существует как минимум два способа получения аннотаций статическим анализом:
%
\begin{itemize*}
\item Метод анализа графа потока управления (CFG);
\item Метод анализа абстрактного синтаксического дерева (AST);
\end{itemize*}
%
К преимуществам статического анализа можно отнести достаточно высокую полноту.
В данном работе было решено остановиться на статическом анализе.

%%%%%%%%%%%%%%%%%%%%%%%%%%%%%%%%%%%%%%%%%%%%%%%%%%%%%%%%%%%%%%%%%%%%%%%%%%%%%%%%
\section{Генерация спецификации}
%%%%%%%%%%%%%%%%%%%%%%%%%%%%%%%%%%%%%%%%%%%%%%%%%%%%%%%%%%%%%%%%%%%%%%%%%%%%%%%%

Непосредственно генерация спецификации является заключительным процессом, во время которого все созданные объекты, которые были извлечены из исходной библиотеки и наполнены, собираются в один файл спецификации.
На данном этапе важно придерживаться определенных правил языка LibSL для того, чтобы генерировать качественную и удобочитаемую для программиста спецификацию. Спецификация также должна соответствовать структуре LibSL:
%
\begin{itemize*}
\item Заголовок библиотеки;
\item Описание подключаемых файлов или пакетов;
\item Описание псевдонимов типов;
\item Описание автоматов;
\item Описание функций API в автоматах;
\end{itemize*}
%

%%%%%%%%%%%%%%%%%%%%%%%%%%%%%%%%%%%%%%%%%%%%%%%%%%%%%%%%%%%%%%%%%%%%%%%%%%%%%%%%
\section{Выводы}
%%%%%%%%%%%%%%%%%%%%%%%%%%%%%%%%%%%%%%%%%%%%%%%%%%%%%%%%%%%%%%%%%%%%%%%%%%%%%%%%

В данном разделе был рассмотрен процесс разработки инструмента генерации формальной спецификации. Этап разработки был разбит на три основных этапа, которые обязательны для выполнения в процессе генерации спецификации.
В совокупности разработанные методы служат в качестве прототипа инструмента генерации спецификации библиотек.