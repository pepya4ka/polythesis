%%%%%%%%%%%%%%%%%%%%%%%%%%%%%%%%%%%%%%%%%%%%%%%%%%%%%%%%%%%%%%%%%%%%%%%%%%%%%%%%
\conclusion
%%%%%%%%%%%%%%%%%%%%%%%%%%%%%%%%%%%%%%%%%%%%%%%%%%%%%%%%%%%%%%%%%%%%%%%%%%%%%%%%

В данной работе были рассмотрены методы, которые могут использоваться для создания инструмента автоматизированной генерации формальной спецификации библиотек, а также инструменты, которые помогают реализовать предложенные методы.

Созданный в ходе работы магистерской диссертации инструмент генерации спецификации позволяет урезать количество времени, затрачиваемое на создание спецификации библиотек, так как предоставляет достаточно полный каркас спецификации с дополнительной информацией, а именно аннотации, описывающие влияния функции на окружаюую среду.
Благодаря использованию абстрактного синтаксического дерева, извлечение аннотаций происходит быстро и точно.

Сгенерированная спецификация может послужить для разработчиков основой спецификаций для разрабатываемых инструментов или библиотек.

Спроектированная система может быть оптимизирована за счет:
%
\begin{itemize*}
\item Извлечения контрактов (предусловий, постусловий и инвариантов) для каждой функции. Добавление контрактов послужит более точному описанию формальной спецификации для библиотеки.
\item Извлечение окружения состояния автоматов и переходов автоматов. Это послужит более точному определению автоматов и их поведению в спецификации библиотеки.
\end{itemize*}
%