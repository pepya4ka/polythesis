
\keywords{
статический анализ,
спецификация,
программные библиотеки
}

\abstractcontent{
Тема выпускной квалификационной работы: «Автоматизация генерации формальной спецификации библиотек».

Основная цель этой магистерской диссертации заключается в разработке процедуры извлечения составляющих формальной спецификации библиотек на языке LibSL.

В данной работе представлена процедура извлечения составляющих спецификации, основана на статическом анализе исходного кода.
Для процедуры извлечения составляющих спецификации была выбрана модель анализа исходного кода, пригодная для разработки инструмента.

Предложенная процедура реализована в прототипе инструмента автоматизированной генерации формальной спецификации библиотеки.
На этапе реализации были рассмотрены возможные инструментарии.
Инструмент был протестирован на различных примерах.
Результаты тестирования показали, что разработанная процедура может использоваться для решения задачи автоматизированной генерации формальной спецификации библиотек на языке LibSL.
}

\keywordsen{
static analysis,
specification,
software library
}

\abstractcontenten{
The topic of the final qualifying work: "Automation of the generation of formal library specifications".

The main purpose of this master's thesis is to develop a procedure for extracting the components of the formal specification of libraries in the LibSL language.

This paper presents a procedure for extracting the components of the specification, based on a static analysis of the source code.
For the procedure of extracting the components of the specification, a source code analysis model suitable for the development of the tool was selected.

The proposed procedure is implemented in a prototype of an automated tool for generating a formal library specification.
At the implementation stage, possible tools were considered.
The tool has been tested on various examples.
The test results showed that the developed procedure can be used to solve the problem of automated generation of formal library specifications in the LibSL language.
}
