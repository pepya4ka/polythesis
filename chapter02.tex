%%%%%%%%%%%%%%%%%%%%%%%%%%%%%%%%%%%%%%%%%%%%%%%%%%%%%%%%%%%%%%%%%%%%%%%%%%%%%%%%
\chapter{Постановка задачи и выбор пути решения}
%%%%%%%%%%%%%%%%%%%%%%%%%%%%%%%%%%%%%%%%%%%%%%%%%%%%%%%%%%%%%%%%%%%%%%%%%%%%%%%%

В данном разделе перечислены задачи, решаемые в ходе работы над данной магистерской диссертацией. Также в данном разделе указаны требования и ограничения, стоящие при выполнении поставленных задач.

%%%%%%%%%%%%%%%%%%%%%%%%%%%%%%%%%%%%%%%%%%%%%%%%%%%%%%%%%%%%%%%%%%%%%%%%%%%%%%%%
\section{Решаемые задачи}
%%%%%%%%%%%%%%%%%%%%%%%%%%%%%%%%%%%%%%%%%%%%%%%%%%%%%%%%%%%%%%%%%%%%%%%%%%%%%%%%

Ниже перечислены поставленные для данной магистерской диссертации задачи:
%
\begin{itemize*}
\item Разработать метод извлечения структуры спецификации, формирования сигнатур функций, типов данных, автоматов и др.;
\item Разработать метод извлечения аннотаций, которые описывают влияние функции на окружающую среду;
\item Убедиться в правильности сгенерированной формальной спецификации на языке LibSL
\end{itemize*}
%

%%%%%%%%%%%%%%%%%%%%%%%%%%%%%%%%%%%%%%%%%%%%%%%%%%%%%%%%%%%%%%%%%%%%%%%%%%%%%%%%
\section{Формулирование требований к разрабатываемой системе}
%%%%%%%%%%%%%%%%%%%%%%%%%%%%%%%%%%%%%%%%%%%%%%%%%%%%%%%%%%%%%%%%%%%%%%%%%%%%%%%%

%%%%%%%%%%%%%%%%%%%%%%%%%%%%%%%%%%%%%%%%
\subsection{Требования к методу извлечения структуры спецификации}
%%%%%%%%%%%%%%%%%%%%%%%%%%%%%%%%%%%%%%%%

Метод извлечения структуры спецификации и формирования сигнатур функций, типов данных и автоматов, должен быть рассчитан на использование с языком Java.

Метод должен корректно обрабатывать код на языке Java 8.

Метод должен иметь возможность работать как с исходным кодом(java-файлами) библиотеки, так и с jar-файлом библиотеки.

В процессе извлечения структуры спецификации не должен извлекать исходный код функций. Процесс касается лишь сигнатур функций, типов данных, автоматов и др.

Метод извлечения структуры спецификации должен исключать все интерфейсы, которые присутствуют в библиотеке, а также те классы, которые не относятся к библиотеке.
Также необходимо исключать из анализа абстрактные классы и методы из библиотеки.

Метод извлечения структуры спецификации должен обрабатывать все основные элементы программы, в которых используется библиотека: возвращаемое методом значение, аргументы метода, типы методов.

Метод извлечения структуры спецификации должен в процессе работы формировать объекты , которые являются элементами LibSL спецификации:
%
\begin{itemize*}
\item Описание псевдонимов типов
\item Описание классов автоматов
\item Описание функций API библиотеки
\end{itemize*}
%

Полученный код инструментом должен быть удобочитаемым для программиста. Например, имена переменных из сигнатуре функции библиотеки должны сохраняться, если это возможно.

%%%%%%%%%%%%%%%%%%%%%%%%%%%%%%%%%%%%%%%%
\subsection{Требования к методу извлечения аннотаций}
%%%%%%%%%%%%%%%%%%%%%%%%%%%%%%%%%%%%%%%%

Разрабатываемой метод должен дополнять структуру спецификации, метод формирования которой описан выше.

Разрабатываемый инструмент должен быть кроссплатформенным и работать под управлением операционных систем

Метод извлечения аннотаций не должен ухудшать сформировнную структуру формальной спецификации библиотеки.

Информация полученая методом извлечения аннотаций должна записываться в определенную аннотацию в языке LibSL.

%%%%%%%%%%%%%%%%%%%%%%%%%%%%%%%%%%%%%%%%
\subsection{Ограничения разрабатываемого инструмента генерации формальной спецификации}
%%%%%%%%%%%%%%%%%%%%%%%%%%%%%%%%%%%%%%%%

Метод извлечения аннотаций должен решать проблему только извлечения информации о влиянии функции на окружающую среду.
То есть, при анализе исходного кода все аргументы метода, которые каким-либо образом изменяются в функции (например, поле объекта одного из аргументов функции асайнится в теле метода на другое значение), будут выделены в отдельную аннотацию в языке LibSL.
Решение проблемы извлечения контрактов (предусловий, постусловий и инвариантов) функции,
остаются за рамками выполнения данной работы, так как выделенные две проблемы являются очень сложными задачами,
поэтому в работе был сделан упор на извлечения аннотаций, которые описывают влияние функции на окружающую среду.

Проблема извлечения окружения состояния автоматов и переходов автоматов по той же причине остается за рамками этой работы.

%%%%%%%%%%%%%%%%%%%%%%%%%%%%%%%%%%%%%%%%%%%%%%%%%%%%%%%%%%%%%%%%%%%%%%%%%%%%%%%%
\section{Анализ задач и выбор пути решения}
%%%%%%%%%%%%%%%%%%%%%%%%%%%%%%%%%%%%%%%%%%%%%%%%%%%%%%%%%%%%%%%%%%%%%%%%%%%%%%%%

Ниже перечислены основные подходы к

%%%%%%%%%%%%%%%%%%%%%%%%%%%%%%%%%%%%%%%%%%%%%%%%%%%%%%%%%%%%%%%%%%%%%%%%%%%%%%%%
\section{Выводы}
%%%%%%%%%%%%%%%%%%%%%%%%%%%%%%%%%%%%%%%%%%%%%%%%%%%%%%%%%%%%%%%%%%%%%%%%%%%%%%%%